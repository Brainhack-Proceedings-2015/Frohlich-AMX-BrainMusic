%% BioMed_Central_Tex_Template_v1.06
%%                                      %
%  bmc_article.tex            ver: 1.06 %
%                                       %

%%IMPORTANT: do not delete the first line of this template
%%It must be present to enable the BMC Submission system to
%%recognise this template!!

%%%%%%%%%%%%%%%%%%%%%%%%%%%%%%%%%%%%%%%%%
%%                                     %%
%%  LaTeX template for BioMed Central  %%
%%     journal article submissions     %%
%%                                     %%
%%          <8 June 2012>              %%
%%                                     %%
%%                                     %%
%%%%%%%%%%%%%%%%%%%%%%%%%%%%%%%%%%%%%%%%%


%%%%%%%%%%%%%%%%%%%%%%%%%%%%%%%%%%%%%%%%%%%%%%%%%%%%%%%%%%%%%%%%%%%%%
%%                                                                 %%
%% For instructions on how to fill out this Tex template           %%
%% document please refer to Readme.html and the instructions for   %%
%% authors page on the biomed central website                      %%
%% http://www.biomedcentral.com/info/authors/                      %%
%%                                                                 %%
%% Please do not use \input{...} to include other tex files.       %%
%% Submit your LaTeX manuscript as one .tex document.              %%
%%                                                                 %%
%% All additional figures and files should be attached             %%
%% separately and not embedded in the \TeX\ document itself.       %%
%%                                                                 %%
%% BioMed Central currently use the MikTex distribution of         %%
%% TeX for Windows) of TeX and LaTeX.  This is available from      %%
%% http://www.miktex.org                                           %%
%%                                                                 %%
%%%%%%%%%%%%%%%%%%%%%%%%%%%%%%%%%%%%%%%%%%%%%%%%%%%%%%%%%%%%%%%%%%%%%

%%% additional documentclass options:
%  [doublespacing]
%  [linenumbers]   - put the line numbers on margins

%%% loading packages, author definitions

\documentclass[twocolumn]{bmcart}% uncomment this for twocolumn layout and comment line below
%\documentclass{bmcart}

%%% Load packages
\usepackage{amsthm,amsmath}
\usepackage{siunitx}
\usepackage{mfirstuc}
%\RequirePackage{natbib}
\usepackage[colorinlistoftodos]{todonotes}
\RequirePackage{hyperref}
\usepackage[utf8]{inputenc} %unicode support
%\usepackage[applemac]{inputenc} %applemac support if unicode package fails
%\usepackage[latin1]{inputenc} %UNIX support if unicode package fails
\usepackage[htt]{hyphenat}

\usepackage{array}
\newcolumntype{L}[1]{>{\raggedright\let\newline\\\arraybackslash\hspace{0pt}}p{#1}}

%%%%%%%%%%%%%%%%%%%%%%%%%%%%%%%%%%%%%%%%%%%%%%%%%
%%                                             %%
%%  If you wish to display your graphics for   %%
%%  your own use using includegraphic or       %%
%%  includegraphics, then comment out the      %%
%%  following two lines of code.               %%
%%  NB: These line *must* be included when     %%
%%  submitting to BMC.                         %%
%%  All figure files must be submitted as      %%
%%  separate graphics through the BMC          %%
%%  submission process, not included in the    %%
%%  submitted article.                         %%
%%                                             %%
%%%%%%%%%%%%%%%%%%%%%%%%%%%%%%%%%%%%%%%%%%%%%%%%%


%\def\includegraphic{}
%\def\includegraphics{}

%%% Put your definitions there:
\startlocaldefs
\endlocaldefs


%%% Begin ...
\begin{document}

%%% Start of article front matter
\begin{frontmatter}

\begin{fmbox}
\dochead{Report from 2015 Brainhack Americas (MX)}

%%%%%%%%%%%%%%%%%%%%%%%%%%%%%%%%%%%%%%%%%%%%%%
%%                                          %%
%% Enter the title of your article here     %%
%%                                          %%
%%%%%%%%%%%%%%%%%%%%%%%%%%%%%%%%%%%%%%%%%%%%%%

\title{Generating music with resting-state fMRI data}
\vskip2ex
\projectURL{Project URL: \url{https://github.com/carolFrohlich/brain-orchestra}}

\author[
addressref={aff1},
corref={aff1},
email={cfrohlich@nki.rfmh.org}
]{\inits{CF} \fnm{C} \snm{Froehlich}}
\author[
addressref={aff3},
%
email={gil.dekel47@myhunter.cuny.edu}
]{\inits{GD} \fnm{Gil} \snm{Dekel}}
\author[
addressref={aff4},
%
email={margulies@cbs.mpg.de}
]{\inits{DSM} \fnm{Daniel S.} \snm{Margulies}}
\author[
addressref={aff1, aff2},
%
email={ccraddock@nki.rfmh.org}
]{\inits{RCC} \fnm{R. Cameron} \snm{Craddock}}

%%%%%%%%%%%%%%%%%%%%%%%%%%%%%%%%%%%%%%%%%%%%%%
%%                                          %%
%% Enter the authors' addresses here        %%
%%                                          %%
%% Repeat \address commands as much as      %%
%% required.                                %%
%%                                          %%
%%%%%%%%%%%%%%%%%%%%%%%%%%%%%%%%%%%%%%%%%%%%%%

\address[id=aff1]{%
  \orgname{Computational Neuroimaging Lab, Center for Biomedical Imaging and
Neuromodulation, Nathan Kline Institute for Psychiatric Research},
  \city{Orangeburg},
  \street{140 Old Orangeburg Rd},
  \postcode{10962},
  \postcode{New York},
  \cny{USA}
}
\address[id=aff2]{%
  \orgname{Center for the Developing Brain, Child Mind Institute},
  \city{New York},
  \street{445 Park Ave},
  \postcode{10022},
  \postcode{New York},
  \cny{USA}
}
\address[id=aff3]{%
  \orgname{City University of New York-Hunter College},
  \city{New York},
  \street{695 Park Ave},
  \postcode{10065},
  \postcode{New York},
  \cny{USA}
}
\address[id=aff4]{%
  \orgname{Max Planck Research Group for Neuroanatomy \& Connectivity, Max Planck
Institute for Human Cognitive and Brain Sciences},
  \city{Leipzig},
  \street{Stephanstraße 1A},
  \postcode{4103},
  \postcode{Leipzig},
  \cny{Germany}
}

%%%%%%%%%%%%%%%%%%%%%%%%%%%%%%%%%%%%%%%%%%%%%%
%%                                          %%
%% Enter short notes here                   %%
%%                                          %%
%% Short notes will be after addresses      %%
%% on first page.                           %%
%%                                          %%
%%%%%%%%%%%%%%%%%%%%%%%%%%%%%%%%%%%%%%%%%%%%%%

\begin{artnotes}
\end{artnotes}

%\end{fmbox}% comment this for two column layout

%%%%%%%%%%%%%%%%%%%%%%%%%%%%%%%%%%%%%%%%%%%%%%
%%                                          %%
%% The Abstract begins here                 %%
%%                                          %%
%% Please refer to the Instructions for     %%
%% authors on http://www.biomedcentral.com  %%
%% and include the section headings         %%
%% accordingly for your article type.       %%
%%                                          %%
%%%%%%%%%%%%%%%%%%%%%%%%%%%%%%%%%%%%%%%%%%%%%%

%\begin{abstractbox}

%\begin{abstract} % abstract
	
%Blank Abstract

%\end{abstract}



%%%%%%%%%%%%%%%%%%%%%%%%%%%%%%%%%%%%%%%%%%%%%%
%%                                          %%
%% The keywords begin here                  %%
%%                                          %%
%% Put each keyword in separate \kwd{}.     %%
%%                                          %%
%%%%%%%%%%%%%%%%%%%%%%%%%%%%%%%%%%%%%%%%%%%%%%

%\vskip1ex

%\projectURL{\url{https://github.com/carolFrohlich/brain-orchestra}}
%\projectURL{https://github.com/carolFrohlich/brain-orchestra}

% MSC classifications codes, if any
%\begin{keyword}[class=AMS]
%\kwd[Primary ]{}
%\kwd{}
%\kwd[; secondary ]{}
%\end{keyword}

%\end{abstractbox}
%
\end{fmbox}% uncomment this for twcolumn layout

\end{frontmatter}

%{\sffamily\bfseries\fontsize{10}{12}\selectfont Project URL: \url{https://github.com/carolFrohlich/brain-orchestra}}

%%% Import the body from pandoc formatted text
\section{Introduction}\label{introduction}

Resting-state fMRI (rsfMRI) data generates time courses with
unpredictable hills and valleys. Peple with musical training may notice
that, to some degree, it resemble the notes of a musical scale.\\Taking
advantage of these similarities, and using only rsfMRI data as input, we
use basic rules of music theory to transform the data into musical form.
Our project is implemented in Python using the
\href{https://code.google.com/p/midiutil/}{midiutil library}.

\section{Approach}\label{approach}

\textit{Data}: We used open rsfMRI from the ABIDE dataset
\cite{di2014autism} preprocessed by the Preprocessed Connectomes Project
\cite{pcp}. We randomly chose 10 individual datasets preprocessed using
C-PAC pipeline \cite{Craddock2013c} with 4 different strategies. To
reduce the data dimensionality, we used the CC200
atlas\textasciitilde{}\cite{cc200} to downsample voxels to 200
regions-of-interest.

\textit{Processing:} The 200 fMRI time courses were analysed to extract
pitch, tempo, and volume--- 3 important attributes for generating music.
For pitch, we mapped the time course amplitudes to Musical Instrument
Digital Interface (MIDI) values in the range of 36 to 84, corresponding
to piano keys within a pentatonic scale. The key of the scale was
determined by the global mean ROI value (calculated across all
timepoints and ROIs) using the equation:
\texttt{(global signal \% 49) + 36}. The lowest tone that can be played
in a certain key was calculated from \texttt{(key \% 12) + 36}. The set
of tones that could be played were then determined from the lowest tone
using a scale. For example, the minor-pentatonic scale's set of tones
was calculated by adding \texttt{[0, 3, 5, 7, 10]} to its lowest tone,
then skipping to the next octave, and then repeating the process until
the value 84 was reached. An fMRI time course was mapped to these
possible tones by scaling its amplitude to the range between the
smallest and largest tones in the set. If a time point mapped to a tone
that was not in the set, it was shifted to the closest allowable tone.
An example of allowed set of tones is shown in Figure \ref{keyfig}.

For tempo, we use the first temporal derivative for calculating the
length of notes, assuming we have 4 lengths (whole, half, quarter and
eighth note). In the time course, if the modulus distance between time
point \textit{t} and \textit{t + 1} was large, we interpreted it as a
fast note (eighth). However, if the distance between \textit{t} and
\textit{t + 1} was close to zero, we assumed it is a slow note (whole).
Using this approach, we mapped all other notes in between.

We used a naive approach for calculating volume in a way that tackles a
problem we had with fast notes: their sound is cut off due to their
short duration. A simple way to solve this is to decrease the volume of
fast notes. Thus, the faster the note, the lower the volume. While a
whole note has volume 100 ({[}0,100{]}), an eighth note has volume 50.

Finally, we selected the brain regions that will play. Users complain
when two similar brain regions play together. Apparently, the brain
produces the same music twice. However, when the regions are distinct,
the music is more pleasant. Thus, we used FastICA \cite{scikitlearn} for
choosing brain regions with maximally uncorrelated time courses.

\section{Results}\label{results}

A framework for generating music from fMRI data, based on music theory,
was developed and implemented as a Python tool yielding several audio
files. When listening to the results, we noticed that music differed
across individual datasets. However, music generated by the same
individual (4 preprocessing strategies) remained similar. Our results
sound different from music obtained in a similar study using EEG and
fMRI data\textasciitilde{}\cite{lu2012scale}.

\section{Conclusions}\label{conclusions}

In this experiment, we established a way of generating music with open
fMRI data following some basic music theory principles. This resulted in
a somewhat naïve but pleasant musical experience. Our results also
demonstrate an interesting possibility for providing feedback from fMRI
activity for neurofeedback experiments.

\begin{figure}
  \includegraphics[width=0.45\textwidth]{figure}
  \caption{\label{keyfig}
  (a)Correspondence between the original time series of one ROI and the generated pitch.
  (b)The first 10 notes of the same ROI as sheet music.
  (c)All possible piano keys the brain can play, from 36 to 84 (in pink).
    We show in red all the possible tones for a C Minor-pentatonic scale, in the range of [36, 84].
    In that case, the lowest key is 36.
    The keys that can be used are: [36, 39, 41, 42, 43, 46, 48, 51, 53, 54, 55, 58, 60, 63, 65, 66, 67, 70, 72, 75, 77, 78, 79, 82, 84]
      }
\end{figure}

%%%%%%%%%%%%%%%%%%%%%%%%%%%%%%%%%%%%%%%%%%%%%%
%%                                          %%
%% Backmatter begins here                   %%
%%                                          %%
%%%%%%%%%%%%%%%%%%%%%%%%%%%%%%%%%%%%%%%%%%%%%%

\begin{backmatter}

\section*{Availability of Supporting Data}
More information about this project can be found at: \url{https://github.com/carolFrohlich/brain-orchestra}. Further data and files supporting this project are hosted in the \emph{GigaScience} repository REFXXX.

\section*{Competing interests}
None

\section*{Author's contributions}
CF wrote the software. GD designed the functions for transforming the
data to midi. DSM pick the algorithm that chooses ROIs, and CF and RCC
wrote the report.

\section*{Acknowledgements}
The authors would like to thank the organizers and attendees of
Brainhack MX.

  
  
%%%%%%%%%%%%%%%%%%%%%%%%%%%%%%%%%%%%%%%%%%%%%%%%%%%%%%%%%%%%%
%%                  The Bibliography                       %%
%%                                                         %%
%%  Bmc_mathpys.bst  will be used to                       %%
%%  create a .BBL file for submission.                     %%
%%  After submission of the .TEX file,                     %%
%%  you will be prompted to submit your .BBL file.         %%
%%                                                         %%
%%                                                         %%
%%  Note that the displayed Bibliography will not          %%
%%  necessarily be rendered by Latex exactly as specified  %%
%%  in the online Instructions for Authors.                %%
%%                                                         %%
%%%%%%%%%%%%%%%%%%%%%%%%%%%%%%%%%%%%%%%%%%%%%%%%%%%%%%%%%%%%%

% if your bibliography is in bibtex format, use those commands:
\bibliographystyle{bmc-mathphys} % Style BST file
\bibliography{./brainhack-report} % Bibliography file (usually '*.bib' )

\end{backmatter}
\end{document}
